% LaTeX Curriculum Vitae Template
%
% Copyright (C) 2004-2009 Jason Blevins <jrblevin@sdf.lonestar.org>
% http://jblevins.org/projects/cv-template/
%
% You may use use this document as a template to create your own CV
% and you may redistribute the source code freely. No attribution is
% required in any resulting documents. I do ask that you please leave
% this notice and the above URL in the source code if you choose to
% redistribute this file.

\documentclass[a4paper]{article}

\usepackage{hyperref}
\usepackage{geometry}

% Comment the following lines to use the default Computer Modern font
% instead of the Palatino font provided by the mathpazo package.
% Remove the 'osf' bit if you don't like the old style figures.
\usepackage[T1]{fontenc}
\usepackage[sc,osf]{mathpazo}
\usepackage{CJK}



% Set your name here
\def\name{Xia Lixun}

% Replace this with a link to your CV if you like, or set it empty
% (as in \def\footerlink{}) to remove the link in the footer:
\def\footerlink{}

% The following metadata will show up in the PDF properties
\hypersetup{
  colorlinks = true,
  urlcolor = black,
  pdfauthor = {\name},
  pdfkeywords = {DC, DSP, ASIC},
  pdftitle = {\name: Curriculum Vitae},
  pdfsubject = {Curriculum Vitae},
  pdfpagemode = UseNone
}

\geometry{
  body={6in, 8.5in},
  left=1.0in,
  top=1.25in
}

% Customize page headers
\pagestyle{myheadings}
\markright{\name}
\thispagestyle{empty}

% Custom section fonts
\usepackage{sectsty}
\sectionfont{\rmfamily\mdseries\Large}
\subsectionfont{\rmfamily\mdseries\itshape\large}

% Other possible font commands include:
% \ttfamily for teletype,
% \sffamily for sans serif,
% \bfseries for bold,
% \scshape for small caps,
% \normalsize, \large, \Large, \LARGE sizes.

% Don't indent paragraphs.
\setlength\parindent{0em}

% Make lists without bullets
\renewenvironment{itemize}{
  \begin{list}{}{
    \setlength{\leftmargin}{1.5em}
  }
}{
  \end{list}
}

\begin{document}
\begin{CJK*}{GBK}{song}

% Place name at left
\textbf{\huge \name \hspace{0.1in} (����ѫ)}

% Alternatively, print name centered and bold:
%\centerline{\huge \bf \name}

\vspace{0.4in}

\begin{minipage}{0.6\linewidth}
  \href{http://maps.baidu.com/#panoid=01002900001405281628351815X&panotype=street&heading=85.57&pitch=-6.99&l=14&tn=B_NORMAL_MAP&sc=0&newmap=1&shareurl=1&pid=01002900001405281628351815X&psp=%7B%22PanoModule%22%3A%7B%22markerUid%22%3A%22cf08edb34ee3c0b52de67b02%22%7D%2C%22PoiInfoPanelModule%22%3A%7B%22uid%22%3A%22cf08edb34ee3c0b52de67b02%22%7D%7D}{Jinghua Apartment (ݼ����Ԣ),} \\
  Nanxieyu Street 18 (������18��),\\
  Suzhou Industry Park (SIP), 215025\\
  Suzhou, Jiangsu, China P.R.\\
  %China P.R.
  %SE-412 96, G\"{o}teborg, \\
  %Sweden
\end{minipage}
\end{CJK*}
\begin{minipage}{0.6\linewidth}
  \begin{tabular}{ll}
    Mobile: & 13013815427 \\
    %Mobile: & (046) 0765838976 \\
    %Fax: &  (919) 962-5678 \\
    Email: & \href{mailto:lixun.xia@outlook.com}{\tt{lixun.xia@outlook.com}} \\
    %Homepage: & \href{http://www.stat-or.unc.edu/}{\tt http://www.stat-or.unc.edu/} \\
  \end{tabular}
\end{minipage}

\vspace{0.3in}

\section*{LinkedIn}
\begin{itemize}
\item \href{http://www.linkedin.com/pub/lixun-xia/1b/212/219}{\tt http://www.linkedin.com/pub/lixun-xia/1b/212/219}
\end{itemize}


\section*{Work Experience}
\begin{description}
  \item[2014.6 - 2015.4 \hspace{0.2in} \textbf{Sr. DSP engineer of Harman, Suzhou, China P.R.}] \hfill \\
  %\vspace{0.1in}
  \begin{itemize}
  \item[--] Responsible for development of vehicle audio amplifiers for Europe and North America markets.
  \item[--] Development and maintenance of Audio Framework.
  \item[--] Cost-down design of amplifiers via novel control algorithm for high-efficient power supply.
  \item[--] Porting and application of Active Noise Cancellation for engine order sound reduction.
  \item[--] Feature application design for amplifiers, for instance, chime beeps etc.
  \item[--] Bug shooting of existing amplifier software.
  \item[--] Documentation of HW/SW design of amplifier products.
  \item[--] Work with system engineers for feature validation and integration.
  \end{itemize}
  \vspace{0.4in}
  \item[2010.10 - 2014.3 \hspace{0.2in}Research and development engineer of Acosense AB, G\"{o}teborg, Sweden] \hfill \\
  \begin{itemize}
  \item[--] Lead the development of active acoustic spectroscopic sensors for fluid measurement.
  \item[--] Determine the specifications for the sensor system with customers directly.
  \item[--] Design of signal capture/conditioning hardware as well as processor circuitry.
  \item[--] Design and implementation of data capture firmware for the FPGA.
  \item[--] Design and implementation of signal processing software for the Linux server.
  \item[--] Implement part of the machine learning code for the Linux server.
  \item[--] SCADA interface and networking software design between sensor terminal and the Linux server.
  \end{itemize}

\end{description}



\section*{Education}
\begin{itemize}
\item 2008.9 - 2010.9 \hspace{0.2in} M.Sc. Integrated Electronic System Design, Chalmers University, Sweden.
\item 2005.9 - 2008.6 \hspace{0.2in} M.Sc. Control Science and Engineering, Central South University, China.
\item 2001.9 - 2005.6 \hspace{0.2in} B.Sc. Automatic Control, Central South University, China.
\end{itemize}



\section*{Project Experience}
\begin{description}
  \item[Audio Framework \hspace{0.2in} 2014 - 2015] \hfill \\
  Audio Framework is the software "Skeleton" defined for any amplifier products on how the audio data are routed and processed. Together with signal processing building blocks as the "muscle", the core functionality of an amplifier emerges. My achievement is, for the first time, a systematic description of the working mechanism of the "skeleton" and "muscle" has been formed within the company. The resultant document is extremely useful for both novice and experienced amplifier design engineers.
  \item[Active Noise Cancellation \hspace{0.2in} 2014 - 2015] \hfill \\
  Active Noise Cancellation is a widely used technique to suppress unwanted noise in an optimal sense via dynamic algorithms. The main purpose is to reduce engine sound orders for quieter cabinet in passenger cars. My achievement is the successful transfer of the project to Suzhou site, build of the application document after code analysis, and creation of a simulation tool for cross-amplifier porting. A verification bench environment has also been set up. It consists of amplifier running ANC algorithm and London BLU-80 DSP running vehicle cabinet emulation algorithm.
  \item[Novel Control of Tracking Power Supply \hspace{0.2in} 2014 - 2015] \hfill \\
  Control of tracking power supply often requires high precision DAC chip for rail indication. My achievement here is the elimination of the DAC chip but utilizing the digital port alone of the DSP chip for rail indication. Cost down has been achieved thanks to this novel method of power rail control in H-class PSU.
  \item[Chime Control via Block Based Processing \hspace{0.2in} 2014 - 2015] \hfill \\
  Chime or beep sounds are often implemented in the rear stages of a fixed point DSP for short time latency in sample-based fashion. Sometimes however, minimal hardware change criterion implies an implementation on block based DSP such as ${\textit SHARC^{TM}}$. The achievement here is such pilot implementation, showing acceptable latency in practice.
  \vspace{0.3in}
  \item[${\textit ACOspector^{TM}}$ -- non-invasive fluid measurement in real-time \hspace{0.2in} 2010 - 2014] \hfill \\
  Lead the development of the industrial fluid property measurement sensor based on active acoustic spectroscopy measurement. Products have been deployed among pulp and paper, chemical factories in Sweden. Some information can be found in \href{http://www.acosense.com}{\tt www.acosense.com}. The concept is based on IoT and cloud computing --- compact, robust and low-cost sensor terminal group paired with one centralized server for data processing in star-topology network. Machine learning running in the backbone builds regression models for fluid property estimation. Hence multiple measurement points can be monitored non-invasively at extremely low cost for modern data-driven factories. As the system designer and developer, my contribution: the path finder of such promising measurement method for the cloud computing age.

  \vspace{0.1in}
  Technical highlight:
  \begin{itemize}
    \item[--] Circuit built from scratch, including design, schematic capture and layout, containing middle-scale FPGA chip for MCU cluster. ADC(PCM4222) for high-dynamic range signal capture. DAC(PCM1794A) for high dynamic signal playback. DDR2 DRAM interfacing with the FPGA as the program space. SPI Flash for system boot load. Tri-port Ethernet switch chip for sensory data transmission with the Linux server via TCP/IP.
    \item[--] All hardware components of the circuitry are carefully determined, taking into account factors like availability, performance and costs.
    \item[--] Realization of a MCU cluster communication protocol for multi-tasking and synchronization based on FIFO queue, rather than using real-time operating system and single MCU core. This allows taking full advantage of the resources of a FPGA chip.
    \item[--] Unit tests for all software components. Integration tests at system level.
    \item[--] Work with members of orthogonal skill sets in agile development style.
    \item[--] All codes and documentations managed by GIT server.
    \item[--] Work with customers in Sweden closely to improve product quality and performance.
    \item[--] Design, prototyping, assembly and testing within our own office and workshop.
    \item[--] Hands-on experience in hardware design for EMI/EMC tests. CE certification verified.
  \end{itemize}
  \vspace{0.3in}
  \item[Matrix Power Converter Design and Implementation \hspace{0.2in} 2007 - 2008] \hfill \\
  Matrix converter can be used as motor drivers(PSM, Induction Motor) thanks to its high power density.
  This is an academic project for my first graduate thesis. Matrix power converter has the merit of higher energy density compared to traditional back-to-back rectifier-inverter topology. Nevertheless the synchronized control of all semiconductor switchers poses harder control problem against the old. In this National Science Foundation project, theory of matrix converters has been studied and a prototype has been engineered. The novel prototype is based on four-leg structure thus allows for more intuitive implementation in a carrier-modulation style.
\end{description}

\section*{Technical Skills}
\begin{itemize}
        \item Modeling, control and stability analysis of power converter for motor drivers.
        \item Modelling and simulation using Matlab/Simulink.
        %\item Digital controller design and implementation.
        %\item Adaptive/optimal filter design (Wiener Filter, LMS/RLS, Kalman filter etc.).
        \item Digital/Statistical signal processing.
        \item Proficient in C/C++.
        \item GCC tool chain for Linux platform.
        \item Microsoft Visual Studio for Windows platform.
        \item Applied knowledge in machine learning (Regression analysis).
        \item Hardware description (VHDL) and verification (Testbench based).
        %\item Digital ASIC backend flow (Cadence Encounter).
        \item Basic knowledge of digital communication.
        \item Shell/Perl scripting language.
        \item Embedded system design and development based upon FPGA/DSP/MCU.
        \item Circuit simulation and analysis with SPICE tools.
        \item Schematic capture, simple PCB layout (Altium Designer/Eagle PCB) and soldering.
        \item Hands-on experience of workbench tools: oscilloscopes, function generator, spectrum analyzer UPV/UPL, AP etc.
        
        \item Knowledge of piezoelectric sensors and actuators.
        \item Hands-on experience of NI/Labview.
        \item Knowledge of EMI/EMC concept and test procedures for product design(Experience in CE certification).
        \item Professional documentation with \LaTeX and Microsoft Office.
        \item Knowledge of Agile Development.
\end{itemize}


\section*{Language Skills}
\begin{itemize}
\item Mandarin: native
\item English: proficient
\item Swedish: basic
\end{itemize}


\section*{Publications}
\begin{itemize}
\item Xia Lixun, Liao Bin, \href{http://publications.lib.chalmers.se/records/fulltext/129177.pdf}{"Hardware Platform For Active Acoustic Spectroscopy Sensors"}, Chalmers University of Technologgy, 2010.
\item Su Mei, Xia Lixun, Sun Yao, et al "Carrier modulation of four-leg matrix converter based on FPGA", Electrical Machines and Systems, 2008. ICEMS 2008. IEEE International Conference on. pp.1247-1250.
\item Yao Sun, Mei Su, Lixun Xia, et al "Randomized carrier modulation for four-leg matrix converter based on optimal Markov chain", Industrial Technology, 2008. ICIT 2008. IEEE International Conference on. pp.1-6.
\item Hengsi Qin, Mei Su, Lixun Xia, et al "A novel controller design method for power converters", IEEE 11th Workshop on Control and Modeling for Power Electronics, 2008. COMPEL 2008. IEEE International Conference on.
\end{itemize}

\section*{Social Services}
\begin{itemize}
\item Translation of the \href{http://www.core.org.cn/OcwWeb/Electrical-Engineering-and-Computer-Science/6-334Spring2003/CourseHome/index.htm}{\emph{6.334 Power Electronics} of Open Courseware(OCW) Project} for \\
China Open Resources for Education(CORE).
\end{itemize}


\section*{MSc. Courses in Chalmers Univ.}
\begin{itemize}
\item DAT091.Introduction to Electronic System Design
\item MCC090.CMOS VLSI Design
\item DAT105.Computer Architecture
\item EDA222.Real Time Systems
\item DAT095.Electronic System Design Project
\item DAT110.EDA Design Methods
\item MVE135.Probability and Random Process With Applications
\item SSY121.Introduction to Communication Engineering
\item SSY125.Digital Communication
\item SSY130.Applied Signal Processing
\item TDA956.Hardware Description and Verification
\item EDA092.Operating System
\item ENM060.Power Electronic Converters
\item TIN092.Algorithms
\end{itemize}

\section*{BSc. Courses in Central South Univ.}
\begin{itemize}
\item C Programming Language
\item Engineering Mathematics
\item Complex Analysis and Integral Transform
\item Physics and Experiments
\item Mathematical Experiment and Modelling
\item Circuit Theory
\item Analog Circuits
\item Digital Circuits
\item Machinery Tool Education
\item Electronics Design Project
\item Electronic Rotating Machinery and Driving
\item Control Theory
\item Computer Architecture and Assembly Language
\item Modern Measurement Technology
\item Process Control and Instruments
\item Fluid Dynamics
\item Computer Control
\item Power Electronics
\item Multi-media Technology
\item PLC and Applications
\item Micro-Controller Unit Technology
\item Driving System Control
\item Computer Simulation
\item Digital Signal Processing
\item Intelligent Control
\end{itemize}


\section*{MSc. Courses in Central South Univ.}
\begin{itemize}
  \item Modern Control Theories
  \item Power Electronics
  \item Matrix Analysis
  \item Functional Analysis and Optimization
  \item System Identification
\end{itemize}


\section*{Driver's License: C1}


\bigskip

% Footer
\begin{center}
  \begin{footnotesize}
    Last updated: \today \\
    \href{\footerlink}{\texttt{\footerlink}}
  \end{footnotesize}
\end{center}

\end{document}
